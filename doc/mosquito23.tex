\documentclass[11pt,a4paper]{article}
\usepackage[utf8]{inputenc}
\usepackage{fullpage}
\usepackage{pdflscape}
\usepackage{amsmath}
\usepackage[margin=0.5cm]{geometry}
\newcommand{\pd}[2]{\frac{\partial #1}{\partial #2}}
\newcommand{\dd}[2]{\frac{\mathrm{d} #1}{\mathrm{d} #2}}
\begin{document}

\section{Equations}

The equations solved by {\tt Mosquito23} are based on the following
\begin{eqnarray*}
\pd{X(a,s,g,m)}{t} &=& -d_{a} X + b_{s,g,m} \left[ \overbrace{X(a-1,s,g,m)}^{a>0} - \overbrace{X}^{a<N} \right] + \overbrace{\nabla \cdot (D_{s,g,m} \nabla X - \mathbf{V}_{s,g,m} X)}^{a=N}
+ \\
a = 0 & & \begin{cases}
\displaystyle \left( \frac{K - \sum_{m^*} \alpha_{m, m^*} \sum_{s,g} \sum_{i<\max(N,1)} X(i,s,g,m^*)}{K} \right) \times \\
\displaystyle \sum_{\substack{g_M, g_F,\\m_M, m_F}}
h(m_M,m_F,m)\ \lambda(g_F, m_F)\ i(g_M, g_F, g)\ \frac{w(m_M, m_F) X(N,M,g_M,m_M)}{\sum_{g^*,m^*} w(m^*, m_F) X(N,M,g^*,m^*)} \times \\
\qquad p(g_M, g_F, m_M, m_F, s) X(N,F,g_F,m_F) 
\end{cases}
\end{eqnarray*}
where $a$ = age ($0$ = newborn, $N$ = adult, $[1, N-1]$ = intermediate stages), 

$s$ = sex ($M$ or $F$), 

$g$ = genotype ($ww$ wildtype, $Gw$ heterozygous GM or $GG$ homozygous GM), 

$m$ = mosquito species ($C$ = \textit{An. coluzzii}, $G$ =\textit{An. gambiae})

and $X(a,s,g,m)$ is abbreviated to $X$.
\section{Parameters}

\begin{center}
\begin{tabular}{ |c|c|c|} 
 \hline
 Parameter name & Explanation & Degrees of freedom \\
 \hline
 $d_{a}$ & Death rate 
 $\begin{cases}
 d_{\textrm{adult}}\textrm{ for }a=N\\
 d_{\textrm{larvae}}\textrm{ for }a<N\\
 \end{cases}$ & 2\\
   \hline
  $b$ & Larval emergence rate & 1\\
  & has distribution $\Gamma(k = N, \theta = b/N)$ & \\
  \hline
  $D$ & Diffusion rate & 1 \\
  \hline
  $\mathbf{V}$ & Advection vector field & Distribution of flight times \\
  \hline
  $K$ & Carrying capacity for species & 1 (for now) \\
	& (spatially explicit, dependent on rainfall)& \\
  \hline
  $h(m_M, m_F, m)$ & Hybridisation rate (following Beeton et al. 2019) & 0\\
  & $ = \begin{cases}
  1 & \textrm{where } m=C \textrm{ and } \{m_M, m_F\}=\{C,C\}  \\
  1 & \textrm{where } m=G \textrm{ and } \{m_M, m_F\}=\{G,G\} \\
  1 & \textrm{where } m=G \textrm{ and either } \{m_M, m_F\}=\{C,G\} \textrm{ or } \{G,C\}  \\
  0 & \textrm{elsewhere}
  \end{cases}$
  & \\
  \hline
  $\lambda$ & Larvae per female & 1 \\
  \hline
  $i(g_M, g_F, g)$ & Inheritance of genotype where $g = \{ww,Gw,GG\}$ & 0 \\
& \begin{tabular}{ |c|c|c|c| } 
 \hline
 & $ww$ & $Gw$ & $GG$ \\
 \hline
 $ww$ & $\{1,0,0\}$ & $\{\frac{1}{2},\frac{1}{2},0\}$ & $\{0,1,0\}$ \\ 
 \hline
 $Gw$ & $\{\frac{1}{2},\frac{1}{2},0\}$   & $\{\frac{1}{4},\frac{1}{2},\frac{1}{4}\}$  & $\{0,\frac{1}{2},\frac{1}{2}\}$  \\ 
 \hline
 $GG$ & $\{0,1,0\}$ & $\{0,\frac{1}{2},\frac{1}{2}\}$  & $\{0,0,1\}$  \\ 
 \hline
\end{tabular}
&  \\
  \hline
    $p(g_M, g_F, m)$ & Proportion of offspring of given sex given genotypes of parents & 1 (accuracy $a$)\\
     & $ = \begin{cases}
		p(Gw, g_F, F) & = \frac{1}{a} - 1 \\
		p(GG, g_F, F) & = \frac{1}{a} - 1 \\
		0.5 & \textrm{elsewhere}
		\end{cases}$ &  \\
    \hline
		$w(m_M, m_F)$ & Relative probability of female of species $m_F$ mating & 1 ($w$) \\
		& with male of species $m_M$ & \\
		& $ = \begin{cases}
		w(C,C) = w(G,G) = 1 \\
		w(C,G) = w(G,C) = w
		\end{cases}$ &  \\
		\hline
\end{tabular}
\end{center}

\section{{\tt Mosquito23}}

{\tt Mosquito23} simplifies the above equations, by assuming that the diffusion and advection is handled by other parts of the code, so that the above equations reduce to a system of ODEs.  To interface with {\tt Mosquito23}, you need to know the ordering of its vector $X$.  For species $m$, genotype $g$, sex $s$ and age $a$, this is
\begin{equation}
X\left(m + gN_{\mathrm{species}} + sN_{\mathrm{species}}N_{\mathrm{genotypes}} + aN_{\mathrm{species}}N_{\mathrm{genotypes}}N_{\mathrm{sexes}} \right) \ ,
\end{equation}
where
\begin{itemize}
\item $N_{\mathrm{species}}$ is the number of species,
\item $N_{\mathrm{genotypes}} = 3$ is the number of genotypes (with ordering $ww=0$, $Gw=1$, and $GG=1$)
\item $N_{\mathrm{sexes}} = 2$ is the number of sexes (with order male$=0$ and female$=1$)
\item $N_{\mathrm{ages}}$ (denoted by $N$ in the above sections) is the number of ages (with ordering newborn$=0$ and adult=$N_{\mathrm{ages}}-1$, with intermediate stages in between these).
\end{itemize}


\subsection{Adults}

All adult populations are goverend by
\begin{equation}
\frac{\mathrm{d}X(N-1, s, g, m)}{\mathrm{d} t} = -d_{\mathrm{adult}}X(N-1, s, g, m) + b X(N-2, s, g, m) \ .
\end{equation}
Here the $N-1$ indicates the adult age bracket, and the $N-2$ is the eldest juvenile age bracket.  If $N=1$ there are no adults, and all populations are governed by the ``Newborn larvae'' equations, below.

\subsection{Intermediate juveniles}

For $0<a<N-1$, the populations are governed by
\begin{equation}
\frac{\mathrm{d}X(a, s, g, m)}{\mathrm{d} t} = -d_{\mathrm{larvae}}X(a, s, g, m) + b \left(X(a-1, s, g, m) - X(a, s, g, m) \right) \ .
\end{equation}
For $N\leq 2$ there are no such intermediate juveniles.

\subsection{Newborn larvae}

For $a=0$, the populations are govened by
\begin{equation}
\frac{\mathrm{d}X(0, s, g, m)}{\mathrm{d} t} = -d_{\mathrm{larvae}}X(0, s, g, m) - b X(a, s, g, m) + B(s, g, m) \ ,
\end{equation}
where
\begin{equation}
B(s, g, m) = \left(1 -  \frac{C(m)}{K} \right) \ldots
\end{equation}
The following terms have been defined in this equation
\begin{itemize}
\item $C(m)$ is the competition that a newborn feels from the rest of the larval populations.  It is
\begin{equation}
C(m) = \sum_{a=0}^{N - 2}\sum_{s=0}^{N_{\mathrm{sexes}} - 1} \sum_{g=0}^{N_{\mathrm{genotypes}} - 1} \sum_{m'=0}^{N_{\mathrm{species}} - 1}\alpha_{m, m'} X(a, s, g, m')\ .
\end{equation}
Notice that this does not include adults $a=N - 1$.  If $N=1$, it is assumed that the carrying-capacity still applies, and the sum over $a$ runs from $0$ to $0$.  The Lotka-Voltera matrix $\alpha$ accounts for inter-specific competition.  It defaults to $a=I$, that is, newborns only feel competition from their own species.
\end{itemize}



\end{document}
