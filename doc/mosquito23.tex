\documentclass[11pt,a4paper]{article}
\usepackage[utf8]{inputenc}
\usepackage{fullpage}
\usepackage{pdflscape}
\usepackage{amsmath}
\usepackage[margin=0.5cm]{geometry}
\newcommand{\pd}[2]{\frac{\partial #1}{\partial #2}}
\newcommand{\dd}[2]{\frac{\mathrm{d} #1}{\mathrm{d} #2}}
\begin{document}

\section{Equations}

The equations solved by {\tt Mosquito23} are based on the following
\begin{eqnarray*}
\pd{X(a,s,g,m)}{t} &=& -d_{a} X + b_{s,g,m} \left[ \overbrace{X(a-1,s,g,m)}^{a>0} - \overbrace{X}^{a<N-1} \right] + \overbrace{\nabla \cdot (D_{s,g,m} \nabla X - \mathbf{V}_{s,g,m} X)}^{a=N-1}
+ \\
a = 0 & & \begin{cases}
\displaystyle \left( \frac{K - \sum_{m^*} \alpha_{m, m^*} \sum_{s,g} \sum_{i<\max(N,1)} X(i,s,g,m^*)}{K} \right) \times \\
\displaystyle \sum_{\substack{g_M, g_F,\\m_M, m_F}}
h(m_M,m_F,m)\ \lambda(g_F, m_F)\ i(g_M, g_F, g)\ \frac{w(m_M, m_F) X(N-1,M,g_M,m_M)}{\sum_{g^*,m^*} w(m^*, m_F) X(N-1,M,g^*,m^*)} \times \\
\qquad p(g_M, g_F, m_M, m_F, s) X(N-1,F,g_F,m_F) 
\end{cases}
\end{eqnarray*}
where $a$ = age ($0$ = newborn, $N-1$ = adult, $[1, N-2]$ = intermediate stages), 

$s$ = sex ($M$ or $F$), 

$g$ = genotype ($ww$ wildtype, $Gw$ heterozygous GM or $GG$ homozygous GM), 

$m$ = mosquito species ($C$ = \textit{An. coluzzii}, $G$ =\textit{An. gambiae})

and $X(a,s,g,m)$ is abbreviated to $X$.


\section{Parameters}

\begin{center}
\begin{tabular}{ |c|c|c|} 
 \hline
 Parameter name & Explanation & Degrees of freedom \\
 \hline
 $d_{a}$ & Death rate 
 $\begin{cases}
 d_{\textrm{adult}}\textrm{ for }a=N-1\\
 d_{\textrm{larvae}}\textrm{ for }a<N-1\\
 \end{cases}$ & 2\\
   \hline
  $b$ & Larval emergence rate & 1\\
  & has distribution $\Gamma(k = N, \theta = b/N)$ & \\
  \hline
  $D$ & Diffusion rate & 1 \\
  \hline
  $\mathbf{V}$ & Advection vector field & Distribution of flight times \\
  \hline
  $K$ & Carrying capacity for species & 1 (for now) \\
	& (spatially explicit, dependent on rainfall)& \\
  \hline
  $h(m_M, m_F, m)$ & Hybridisation rate (following Beeton et al. 2019) & 0\\
  & $ = \begin{cases}
  1 & \textrm{if } m=m_{M}=m_{F} \\
  0 & \textrm{otherwise}
  \end{cases}$
  & (may be set in code)\\
  \hline
  $\lambda$ & Larvae per female & 1 \\
  \hline
  $i(g_M, g_F, g)$ & Inheritance of genotype where $g = \{ww,Gw,GG\}$ & 0 \\
& \begin{tabular}{ |c|c|c|c| } 
 \hline
 & $ww$ & $Gw$ & $GG$ \\
 \hline
 $ww$ & $\{1,0,0\}$ & $\{\frac{1}{2},\frac{1}{2},0\}$ & $\{0,1,0\}$ \\ 
 \hline
 $Gw$ & $\{\frac{1}{2},\frac{1}{2},0\}$   & $\{\frac{1}{4},\frac{1}{2},\frac{1}{4}\}$  & $\{0,\frac{1}{2},\frac{1}{2}\}$  \\ 
 \hline
 $GG$ & $\{0,1,0\}$ & $\{0,\frac{1}{2},\frac{1}{2}\}$  & $\{0,0,1\}$  \\ 
 \hline
\end{tabular}
&  \\
  \hline
    $p(g_M, g_F, s)$ & Proportion of offspring of given sex given genotypes of parents & 1 (accuracy $a$)\\
		& $ = \begin{cases}
		p(Gw, g_F, M) = p(GG, g_F, M) & = a \\
		p(Gw, g_F, F) = p(GG, g_F, F) & = 1 - a \\
		p(ww, Gw, F) = p(ww, GG, F) & = f \\
		p(ww, Gw, M) = p(ww, GG, M) & = 1 - f \\
		0.5 &\textrm{elsewhere}
		\end{cases}$ & \\
    \hline
		$w(m_M, m_F)$ & Relative probability of female of species $m_F$ mating & 1 ($w$) \\
		& with male of species $m_M$ & (defaults to $w=0$) \\
		& $ = \begin{cases}
		w(C,C) = w(G,G) = 1 \\
		w(C,G) = w(G,C) = w
		\end{cases}$ &  \\
		\hline
$\alpha(m, m')$ & Lotka-Volterra competition between species & Defaults to $\alpha=I$ \\
 & & (may be set in code)
\\
\hline
\end{tabular}
\end{center}

\section{{\tt Mosquito23}}

{\tt Mosquito23} simplifies the above equations, by assuming that the diffusion and advection is handled by other parts of the code, so that the above equations reduce to a system of ODEs.  To interface with {\tt Mosquito23}, you need to know the ordering of its vector $X$.  For species $m$, genotype $g$, sex $s$ and age $a$, this is
\begin{equation}
X\left(m + gN_{\mathrm{species}} + sN_{\mathrm{species}}N_{\mathrm{genotypes}} + aN_{\mathrm{species}}N_{\mathrm{genotypes}}N_{\mathrm{sexes}} \right) \ ,
\end{equation}
where
\begin{itemize}
\item $N_{\mathrm{species}}$ is the number of species,
\item $N_{\mathrm{genotypes}} = 3$ is the number of genotypes (with ordering $ww=0$, $Gw=1$, and $GG=2$)
\item $N_{\mathrm{sexes}} = 2$ is the number of sexes (with order male$=0$ and female$=1$)
\item $N_{\mathrm{ages}}$ (denoted by $N$ in the above sections) is the number of ages (with ordering newborn$=0$ and adult=$N_{\mathrm{ages}}-1$, with intermediate stages in between these).
\end{itemize}
The next sections write the equations explicitly and add some explanation.

\subsection{Adults}

All adult populations are goverend by
\begin{equation}
\frac{\mathrm{d}X(N-1, s, g, m)}{\mathrm{d} t} = -d_{\mathrm{adult}}X(N-1, s, g, m) + b X(N-2, s, g, m) \ .
\end{equation}
Here $N-1$ indicates the adult age bracket, and the $N-2$ is the eldest juvenile age bracket.  The first term on the right-hand side describes the mortality of adults, while the second term desribes aging from the eldest juveniles.  If $N=1$ there are no adults, and all populations are governed by the ``Newborn larvae'' equations, below.

\subsection{Intermediate juveniles}

For $0<a<N-1$, the populations are governed by
\begin{equation}
\frac{\mathrm{d}X(a, s, g, m)}{\mathrm{d} t} = -d_{\mathrm{larvae}}X(a, s, g, m) + b \left(X(a-1, s, g, m) - X(a, s, g, m) \right) \ .
\end{equation}
For $N\leq 2$ there are no such intermediate juveniles.  The first term on the right-hand side describes the mortality of this age-bracket of juveniles, while the term involving $b$ describes aging to/from older/younger age brackets

\subsection{Newborn larvae}

For $a=0$, the populations are govened by
\begin{equation}
\frac{\mathrm{d}X(0, s, g, m)}{\mathrm{d} t} = -d_{\mathrm{larvae}}X(0, s, g, m) - b X(0, s, g, m) + B(s, g, m) \ .
\end{equation}
The first term describes mortality of newborns, while the second describes aging into the next age-bracket of juveniles.  The final term describes the birth of newborn larvae.  It is
\begin{equation}
B(s, g, m) = L\left(1 -  \frac{C(m)}{K} \right) \sum_{g_{M}, g_{F}, m_{M}, m_{F}}P_{\mathrm{offspring}}(s, g, m | g_{M}, g_{F}, m_{M}, m_{F}) P_{\mathrm{mating}}(g_{M}, m_{M}, m_{F})\lambda X(N-1, F, g_{F}, m_{F})
\end{equation}
This equation deserves explanation.
\begin{itemize}
\item $C(m)$ is the competition that a newborn feels from the rest of the larval populations.  It is
\begin{equation}
C(m) = \sum_{a=0}^{N - 2}\sum_{s=0}^{N_{\mathrm{sexes}} - 1} \sum_{g=0}^{N_{\mathrm{genotypes}} - 1} \sum_{m'=0}^{N_{\mathrm{species}} - 1}\alpha_{m, m'} X(a, s, g, m')\ .
\end{equation}
Notice that this does not include adults $a=N - 1$.  If $N=1$, it is assumed that the carrying-capacity still applies, and the sum over $a$ runs from $0$ to $0$.  The Lotka-Voltera matrix $\alpha$ accounts for inter-specific competition.  It defaults to $a=I$, that is, newborns only feel competition from their own species.  There is one further caveat: if $K<K_{\mathrm{min}}$ for user-defined $K_{\mathrm{min}}$ (which defaults to $10^{-6}$) then $B=0$ for all $s$, $g$ and $m$.  This helps with numerical stability in the case when $K$ is time-dependent.
\item The function $L(x)=0$ if $x\leq 0$, while $L(x)=x$ for $x>0$.  This is to ensure that if $C(m)>K$ no newborns are produced.
\item $X(N-1, F, g_{F}, m_{F})$ is the number of adult ($a=N-1$), females of genotype $g_{F}$ and species $m_{F}$.  So $\lambda X(N-1, F, g_{F}, m_{F})$ is the number of newborns produced by these female per timestep.
\item $P_{\mathrm{mating}}(g_{M}, m_{M}, m_{F})$ is the probability that a male adult of genogype $g_{M}$ and species $m_{M}$ successfully mates with a female adult of species $m_{F}$ to produce newborn.  It is
\begin{equation}
P_{\mathrm{mating}}(g_{M}, m_{M}, m_{F}) = \frac{w(m_{M}, m_{F})X(N-1, M, g_{M}, m_{M})}{\sum_{g'=0}^{N_{\mathrm{genotypes}}}\sum_{m'=0}^{N_{\mathrm{species}}}w(m', m_{F})X(N-1, M, g', m')} \ .
\end{equation}
The numerator is the number of matings between male of species $m_{M}$ and genotype $g_{M}$ and the female, while the denominator normalises the probability.  The matrix $w$ defaults to the identity.
\item $P_{\mathrm{offspring}}(s, g, m | g_{M}, g_{F}, m_{M}, m_{F})$ is the probability the offspring will have sex $s$, genotype $g$ and species $m$, given the genotypes and species of its parents.  This is
\begin{equation}
P_{\mathrm{offspring}}(s, g, m | g_{M}, g_{F}, m_{M}, m_{F}) = h(m_{M}, m_{F}, m)i(g_{M}, g_{F}, g)p(g_{M}, g_{F}, m_{M}, m_{F}, s) \ .
\end{equation}
The first term, $h$, determines the hybridisation between species, the second determines the inheritance of genotypes, while the final term describes any sex bias in the offspring.  The hybridisation defaults to $h=1$ if $m_{M}=m_{F}=m$ and zero otherwise.  
\item Finally, these expressions are summed over all possible parental genotypes and species using $\sum_{g_{M}, g_{F}, m_{M}, m_{F}}$.
\end{itemize}

\subsection{Time integration}

The ODEs in {\tt Mosquito23} may be integrated in time using one of the following methods.
\begin{enumerate}
\item Explicit-Euler, where $X(t + \Delta t) = \Delta t f(X(t))$.  This is fast, but results in the greatest error.
\item Runge-Kutta4, where $X(t + \Delta t)$ is given by the fourth-order Runge-Kutta formula.  This is approximately 4 times slower than explicit-Euler.
\item Scipy's {\tt solve\_ivp} method.  This is over 100 times slower than explicit-Euler, but is the most accurate.
\end{enumerate}

In addition, adaptive time-stepping is the default.  Here, the user defines $\Delta t$, and if the algorithm detects that any $X(t + \Delta t) < 0$, the time-step is solved using a number of smaller sub-time-steps, chosen to guarantee that all $X$ remain non-negative.  This type of behaviour occurs when the time-dependent carrying capacity suddenly reduces, and the explicit-Euler or Runge-Kutta4 methods produce large negative changes in population numbers, which, if allowed, would result in $X<0$.  Solving the problem using smaller sub-time-steps overcomes this problem.  In this algorithm, there is a minimum $\Delta t$ allowed, which defaults to $10^{-12}$, below which the algorithm exits with an error.

Finally, a user-defined cutoff, $c$, is placed on $X(t + \Delta t)$.  If $X(t + \Delta t) < c$ (at the end of a time step) then $X(t + \Delta t)$ is set to zero.  This prevents anomalous round-off and precision errors from accumulating.  The default value of $c$ is $10^{-6}$.

\section{\tt Mosquito23F}
\subsection{Fecundity limiting (old version)}
Female-sexed eggs of either $Gw$ or $GG$ fathers are assumed to become mostly inviable after fertilization, such that the sex ratio is skewed male with proportion $a$. The number of male-sexed eggs is assumed to stay the same as for wildtype mosquitoes.

\begin{equation}
 p(g_M, g_F, s) = \begin{cases}
		p(Gw, g_F, F) = p(GG, g_F, F) & = \frac{1}{2}\left(\frac{1}{a} - 1\right) \\
		0.5 & \textrm{elsewhere}
		\end{cases}
\end{equation}

\subsection{Fecundity preserving (current version)}
Female-sexed sperm of either $Gw$ or $GG$ fathers are assumed to become mostly inviable before implantation in eggs. In this case, the \emph{total} number of eggs remains the same as for wildtype mosquitoes (as the eggs are not affected by the construct), but the sex ratio is skewed male with proportion $a > 0.5$.

We also include a corresponding female bias $b > 0.5$ in offspring of female $Gw$ or $GG$ ``survivors'' with wildtype males as demonstrated in Galizi et al. (2014) (Supp Table 6) (TODO: there is probably also an overall decrease in fecundity in this case that we should possibly also model, with $66.0 \pm 3.8$ eggs hatching versus $79.4 \pm 2.2$ for the control).

\begin{equation}
 p(g_M, g_F, s) = \begin{cases}
		p(Gw, g_F, M) = p(GG, g_F, M) & = a \\
		p(Gw, g_F, F) = p(GG, g_F, F) & = 1 - a \\
		p(ww, Gw, F) = p(ww, GG, F) & = f \\
		p(ww, Gw, M) = p(ww, GG, M) & = 1 - f \\
		0.5 &\textrm{elsewhere}
		\end{cases}
\end{equation}



\end{document}
